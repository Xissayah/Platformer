%% LyX 2.3.7 created this file.  For more info, see http://www.lyx.org/.
%% Do not edit unless you really know what you are doing.
\documentclass[english]{article}
\usepackage[utf8]{inputenc}

\makeatletter

%%%%%%%%%%%%%%%%%%%%%%%%%%%%%% LyX specific LaTeX commands.
\InputIfFileExists{t2aenc.def}{}{%
  \errmessage{File `t2aenc.def' not found: Cyrillic script not supported}}
\DeclareRobustCommand{\cyrtext}{%
  \fontencoding{T2A}\selectfont\def\encodingdefault{T2A}}
\DeclareRobustCommand{\textcyr}[1]{\leavevmode{\cyrtext #1}}


\makeatother

\usepackage{babel}
\begin{document}
\title{Отчет по игре-платформеру для зачета по программированию}
\maketitle
\begin{enumerate}
\item Создание персонажа
\begin{itemize}
\item Написан код для движения и прыжков, ограничения по высоте (проигрыш
при падении), применения анимации
\item С помощью Узла AnimatedSprite2D добавлены анимации Простоя, Прыжка
и Бега
\item Узлом CollisionShape2D добавлена коллизия (2 слоя: 1 - player, 2 -
environment)
\item Создана камера (Узел - Camera2D)
\item К камере привязан ParallaxBackground с 4 картинками фона, которые
движутся с разной скоростью
\item В код добавлены кнопки для перезапуска игры (R) и выхода (Esc)
\end{itemize}
\item Создание игрового уровня
\begin{itemize}
\item На уровень был добавлен Персонаж (+фон) и Узел TileMap
\item С помощью TileMap были созданы платформы с коллизией по которым перемещается
персонаж
\item При помощи Узла Area2D была создана дверь, касание которой является
условием перехода на новый уровень
\begin{itemize}
\item сигнал body\_entered вызывает смену сцены на 2-ой уровень (Level2.tscn)
\end{itemize}
\item Также была добавлена фоновая музыка
\end{itemize}
\item Создание экранов Победы и Поражения
\begin{itemize}
\item Оба экрана сделаны по единому принципу, различаются лишь тексты на
них и условия появления
\item Был добавлен Узел Control
\item К нему подключен TextureRect с фоном и CenterContainer (чтобы кнопки
и текст были в 1 месте)
\item К Контейнеру привязан VBoxContainer с Label'ом и HBoxContainer'ом
\item К HBoxContainer'у привязаны 2 кнопки ``RESTART'' и ``EXIT'' сигналы
которых перезапускают игру (переходят к сцене ``Level1.tscn'') или
выходят из игры соответственно
\end{itemize}
\item Создание врагов (шипы)
\begin{itemize}
\item Создан узел Area2D со спрайтом шипов и коллизией
\item К сигналу узла добавлен скрипт Поражения (переход к сцене ``GameOver.tscn'')
\item Шипы были расставлены на уровнях
\end{itemize}
\item Создание 2-го уровня
\begin{itemize}
\item Копипаст из 1-го уровня Tilemap'a, двери, персонажа и шипов
\item При помощи Tilemap'a построен 2-ой уровень, добавлены шипы и дверь
\item При касании шипов или двери игра запускается заново (с 1-го уровня),
двери - Игрок побеждает (переключение сцены на ``youWin.tscn'')
\end{itemize}
\end{enumerate}
P. S. Музыка перенесена из Level1.tscn в Player.tscn
\end{document}
